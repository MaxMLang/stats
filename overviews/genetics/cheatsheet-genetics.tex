\documentclass{article}
\usepackage{graphicx} % Required for inserting images
\usepackage{amsmath}
\usepackage{amsfonts}
\usepackage{amssymb}

\title{Simulation Methods}
\author{Max Lang}
\date{January 2024}

\begin{document}

\maketitle


\section{W-F Model Time back until coalescent, limiting process}
In a Wright-Fisher model with population size $M$, for a sample of size two taken from the population define $\tau_2$ as the time back until a coalescence event occurs. Then setting $T_2=\tau_2 / M$ to measure time in units of $M$ generations, in the limit as $M \rightarrow \infty, T_2$ has an exponential distribution: $T_2^{\sim} \exp (1)$

\subsection{Proof}
Proof From above:
$$
\begin{aligned}
P\left(\tau_2\right. & \leq k)=1-P\left(\tau_2>k\right) \\
& =1-\left(1-\frac{1}{M}\right)^k \\
P\left(T_2\right. & \leq t)=P\left(\tau_2 \leq M t\right)=P\left(\tau_2 \leq\lfloor M t\rfloor\right) \\
& =1-\left(1-\frac{1}{M}\right)^{\lfloor M t\rfloor}=\left(1-P\left(\tau_2 +
\lfloor M t\rfloor\right)\right. \\
& \rightarrow 1-e^{-t} \text { as } M \rightarrow \infty,
\end{aligned}
$$
for any $t>0$ : the c.d.f of $\exp (1)$


\section{Coalescent Definition }
The coalescent is a distribution on binary trees. Starting with $n$ lineages, pairs of lineages coalesce backward in time until a single common ancestor is reached. Defining times $T_n, T_{n-1}, \ldots, T_2$ while $n, n-1, \ldots, 2$ ancestors remain, the times $T_j$ are independent and exponentially distributed:
$$
T_j \sim \exp \left[\binom{j}{2}\right] \quad f_j(t)=\binom{j}{2} e^{-\binom{j}{2}^t}, \mathrm{t}>0 \quad E\left(T_j\right)=\frac{2}{j(j-1)}
$$

At the time of coalescence from $j$ to $j-1$ lineages, a pair of lineages is chosen at random from the $j(j-1) / 2$ possibilities and coalesces.

\section{TMRCA}

$$
W_n=T_n+T_{n-1}+\ldots+T_2
$$
\subsection{Mean and variance of the TMRCA Immediately:}
$$
\begin{aligned}
& E\left(W_n\right)=E\left(T_n\right)+E\left(T_{n-1}\right)+\ldots+E\left(T_2\right) \\
&=\frac{2}{n(n-1)}+\frac{2}{(n-1)(n-2)}+\ldots+\frac{2}{2 \times 1} \\
& =\sum_{j=2}^n \frac{2}{j(j-1)}=\sum_{j=2}^n \frac{2}{(j-1)}-\sum_{j=2}^n \frac{2}{j} \\
&  \\
&=2\left(1-\frac{1}{n}\right)
\end{aligned}
$$


\end{document}