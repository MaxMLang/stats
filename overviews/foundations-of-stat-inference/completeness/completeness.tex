\documentclass{article}
\usepackage{graphicx} % Required for inserting images

\title{Completeness of a statistical model and statistic Intuition}


\begin{document}

\maketitle
\section{Completeness of a statistical model and statistic}
The provided definition deals with the concept of "completeness" in a statistical model and its relevance to a statistic used within that model. Here's an intuitive explanation:
\newline
Imagine you have a box full of differently colored balls, and these balls represent different possible outcomes or observations in an experiment or study. The color of a ball is determined by some underlying truth or parameter (denoted by $\theta$ in your definition). A statistical model is a collection of all the possible ways (probability distributions) you might reach into the box and pull out a ball, depending on what the true parameter $\theta$ is.
\newline
Now, a statistical model is said to be "complete" if, whenever you use a particular function \( h \) to look at your balls (observations), and on average (expectation \( E \)), this function tells you nothing (comes out to zero) regardless of which color of ball (parameter $\theta$) is the true one, then this function \( h \) must actually be the 'nothing' function; that is, it gives you zero for every single ball (observation) in your box.
\newline
Transferring this to a statistic \( T \), a statistic is "complete" if, whenever you use a particular function \( h \) to look at your statistic and on average it tells you nothing, then this function must actually be the 'nothing' function. This means that the statistic \( T \) is capturing all the information about the underlying parameter $\theta$ that is available in the data. If \( T \) were not complete, there would be some other function of the data that could tell us more about $\theta$, which would mean \( T \) is missing some information about the true state of things.
\newline
In other words, a complete statistical model or statistic leaves no leftover information that could be exploited to learn more about the parameter $\theta$. It's like saying that the model or statistic is perfectly tailored to the parameter it is estimating; there's no more "juice to squeeze" from the data.

\end{document}