\documentclass{article}
\usepackage{amsmath, amssymb}

\begin{document}

\section*{Logit Logistic Regression}

\textbf{Mathematical Basis:}
\begin{itemize}
  \item The Logit model uses the logistic function to map probabilities. The formula is: 
  \[ P(Y=1) = \frac{1}{1 + e^{-(\beta_0 + \beta_1 X)}} \]
  Here, $P(Y=1)$ is the probability of the event occurring (Y=1), $\beta_0$ is the intercept, $\beta_1$ is the coefficient for the predictor $X$, and $e$ is the base of natural logarithms.
\end{itemize}

\textbf{Interpretation:}
\begin{itemize}
  \item The coefficients in a logit model represent the log odds. For a one-unit increase in $X$, the log odds of $Y=1$ increase by $\beta_1$.
  \item To interpret the coefficients intuitively, we often convert log odds to odds ratios by exponentiating the coefficients.
\end{itemize}

\textbf{Marginal Effects:}
\begin{itemize}
  \item While the coefficients give us the change in log odds, they don't directly tell us how the probability of $Y=1$ changes with a unit change in $X$. That's where marginal effects come in, giving us a more intuitive understanding of the impact of predictor variables.
\end{itemize}

\section*{Probit Regression}

\textbf{Mathematical Basis:}
\begin{itemize}
  \item The Probit model uses the cumulative distribution function (CDF) of the standard normal distribution. The formula is:
  \[ P(Y=1) = \Phi(\beta_0 + \beta_1 X) \]
  Here, $\Phi$ denotes the CDF of the standard normal distribution.
\end{itemize}

\textbf{Interpretation:}
\begin{itemize}
  \item In Probit, the coefficients represent the change in the z-score of the standard normal distribution that corresponds to a one-unit change in $X$.
  \item The interpretation of these coefficients is less intuitive than in the Logit model, as they relate to changes in the z-score rather than odds.
\end{itemize}

\textbf{Marginal Effects:}
\begin{itemize}
  \item Marginal effects are particularly important in Probit models because the coefficients themselves are difficult to interpret directly. The marginal effect in Probit tells us how the probability of $Y=1$ changes with a small change in $X$, and it varies depending on the values of $X$.
\end{itemize}

\section*{Comparison and When to Use Which}

\textbf{Differences in Interpretation:}
\begin{itemize}
  \item Logit coefficients are interpreted in terms of log odds and are more intuitive. Probit coefficients are interpreted in terms of z-scores of a normal distribution and are less intuitive.
  \item Marginal effects are essential for both, but more so for Probit due to the less intuitive nature of its coefficients.
\end{itemize}

\textbf{Differences Error Distribution:}
\begin{itemize}
    \item \textbf{Logit Model Error Distribution: Logistic Distribution}
    \begin{itemize}
        \item Results in a logistic function for the probability model.
    \end{itemize}
    \begin{itemize}
        \item Assumes a slightly different distribution of error terms compared to Probit, with more weight in the tails.
    \end{itemize}
    \textbf{\item Probit Model Error Distribution: Standard Normal Distribution}
    \begin{itemize}
        \item Aligns with many traditional statistical assumptions of normality.
    \end{itemize}
    \begin{itemize}
        \item Uses the CDF of the standard normal distribution for the probability model.
    \end{itemize}
\end{itemize}

\textbf{When to Use Probit:}
\begin{itemize}
  \item Probit is often preferred when the underlying latent process being modeled is believed to follow a normal distribution, as it aligns with the Probit model's assumption of a normal error term.
  \item It's also preferred in some disciplines, like economics, where the normal CDF is a standard modeling choice.
\end{itemize}

\textbf{When to Use Logit:}
\begin{itemize}
  \item Logit is preferred when the focus is on interpreting the coefficients in terms of odds ratios, which are more intuitive.
  \item It's also used widely in fields like medicine and public health.
\end{itemize}

\end{document}
